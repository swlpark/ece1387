\documentclass[12pt]{article}

\usepackage[letterpaper, hmargin=0.75in, vmargin=0.75in]{geometry}
\usepackage{float}

% Fill in these values to make your life easier
\newcommand{\iterations}{800000000}
\newcommand{\physicalcores}{4}
\newcommand{\virtualcpus}{4}

\pagestyle{empty}

\title{ECE 459: Programming for Performance\\Assignment 1}
\author{Lawrence Park}
\date{January 30, 2012}

\begin{document}

\maketitle

\section*{Part 2 - Benchmarking}

These experiments were run on a Intel(R) Core(TM) i5-3570 CPU. It has \physicalcores{} physical cores and \virtualcpus{} virtual
CPUs.

\begin{table}[H]
  \centering
  \begin{tabular}{lr}
    & {\bf Time (s)} \\
    \hline
    Run 1 & 11.041 \\
    Run 2 & 11.061 \\
    Run 3 & 11.05 \\
    Run 4 & 11.058 \\
    Run 5 & 11.06 \\
    Run 6 & 11.07\\
    \hline
    Average & 11.06 \\
  \end{tabular}
  \caption{Benchmark results for sequential execution ({\tt i} = \iterations{})}
  \label{tbl_sequential}
\end{table}

Refer to {\bf Table~\ref{tbl_sequential}} and estimate runtime with
\physicalcores{} physical cores.
\\
\indent
Assuming the code is 100\% parallel, the speedup is calculated: S = 1 / 1/N = 4.
Therefore, the parallel runtime is estimated to be Ts / S = 2.765 s.

\begin{table}[H]
  \centering
  \begin{tabular}{lr}
    & {\bf Time (s)} \\
    \hline
    Run 1 & 2.938 \\
    Run 2 & 2.935 \\
    Run 3 & 2.934 \\
    Run 4 & 2.927 \\
    Run 5 & 2.927 \\
    Run 6 & 2.953 \\
    \hline
    Average & 2.94 \\
  \end{tabular}
  \caption{Benchmark results for parallel execution ({\tt i} = \iterations{},
    {\tt t} = \physicalcores{})}
  \label{tbl_parallel_physicalcores}
\end{table}

Refer to {\bf Table~\ref{tbl_parallel_physicalcores}}, does this agree with your
predicted runtime? Write your answer here.
\\
\indent
Speedup with pthread execution is calculated to be approximately 3.76 which is fairly close to our crude speedup approximation of 4.
The difference can be explained by the fact that not all executed code was parallel, and there exists an overhead for creating pthreads and joining them.
Our original sequential algorithm for Monte Carlo approximation is "embarrassingly parallel," and it explains why predicted speedup is close to the actual speedup.

\begin{table}[H]
  \centering
  \begin{tabular}{lr}
    & {\bf Time (s)} \\
    \hline
    Run 1 & 2.95 \\
    Run 2 & 2.976 \\
    Run 3 & 2.962 \\
    Run 4 & 2.97 \\
    Run 5 & 2.966 \\
    Run 6 & 2.953 \\
    \hline
    Average & 2.962 \\
  \end{tabular}
  \caption{Benchmark results for parallel execution ({\tt i} = \iterations{},
    {\tt t} = \virtualcpus{})}
  \label{tbl_parallel_virtualcpus}
\end{table}

Refer to {\bf Table~\ref{tbl_parallel_virtualcpus}} calculate the speedup, and
verify it is less than \virtualcpus{}.
\\
\indent
Intel(R) Core(TM) i5-3570 CPU does not have hyperthreading, and its virtual CPUs are of the same number as its physical cores.
Speedup, therefore, is estimated at 4 (i.e. number of physical cores), and the actual speedup was 3.73.


\begin{table}[H]
  \centering
  \begin{tabular}{lr}
    & {\bf Time (s)} \\
    \hline
    Run 1 & 3.2 \\
    Run 2 & 3.054 \\
    Run 3 & 3.049 \\
    Run 4 & 3.423 \\
    Run 5 & 2.961 \\
    Run 6 & 3.058 \\
    \hline
    Average & 3.124 \\
  \end{tabular}
  \caption{Benchmark results for parallel execution ({\tt i} = \iterations{},
    {\tt t} = \virtualcpus{} + 1)}
  \label{tbl_parallel_virtualcpus_plus_one}
\end{table}

Refer to {\bf Table~\ref{tbl_parallel_virtualcpus_plus_one}}, calculate the
speedup and compare it to {\bf Table~\ref{tbl_parallel_virtualcpus}} (or
{\bf Table~\ref{tbl_parallel_physicalcores}} if you don't have hyperthreading),
which performs better? Write your explanation here.
\\
\indent
The speedup with 5 threads was 3.53 which is slower than the speedup achieved with 4 threads which was 3.76. This difference makes sense because with 5 threads, the CPU cores will not have even amount
of workload as they did with 4 threads. With 5 threads created, one of the 4 cores will have to execute the last thread while the other cores stay idle.

\end{document}
